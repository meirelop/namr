
% Default to the notebook output style

    


% Inherit from the specified cell style.




    
\documentclass[11pt]{article}

    
    
    \usepackage[T1]{fontenc}
    % Nicer default font (+ math font) than Computer Modern for most use cases
    \usepackage{mathpazo}

    % Basic figure setup, for now with no caption control since it's done
    % automatically by Pandoc (which extracts ![](path) syntax from Markdown).
    \usepackage{graphicx}
    % We will generate all images so they have a width \maxwidth. This means
    % that they will get their normal width if they fit onto the page, but
    % are scaled down if they would overflow the margins.
    \makeatletter
    \def\maxwidth{\ifdim\Gin@nat@width>\linewidth\linewidth
    \else\Gin@nat@width\fi}
    \makeatother
    \let\Oldincludegraphics\includegraphics
    % Set max figure width to be 80% of text width, for now hardcoded.
    \renewcommand{\includegraphics}[1]{\Oldincludegraphics[width=.8\maxwidth]{#1}}
    % Ensure that by default, figures have no caption (until we provide a
    % proper Figure object with a Caption API and a way to capture that
    % in the conversion process - todo).
    \usepackage{caption}
    \DeclareCaptionLabelFormat{nolabel}{}
    \captionsetup{labelformat=nolabel}

    \usepackage{adjustbox} % Used to constrain images to a maximum size 
    \usepackage{xcolor} % Allow colors to be defined
    \usepackage{enumerate} % Needed for markdown enumerations to work
    \usepackage{geometry} % Used to adjust the document margins
    \usepackage{amsmath} % Equations
    \usepackage{amssymb} % Equations
    \usepackage{textcomp} % defines textquotesingle
    % Hack from http://tex.stackexchange.com/a/47451/13684:
    \AtBeginDocument{%
        \def\PYZsq{\textquotesingle}% Upright quotes in Pygmentized code
    }
    \usepackage{upquote} % Upright quotes for verbatim code
    \usepackage{eurosym} % defines \euro
    \usepackage[mathletters]{ucs} % Extended unicode (utf-8) support
    \usepackage[utf8x]{inputenc} % Allow utf-8 characters in the tex document
    \usepackage{fancyvrb} % verbatim replacement that allows latex
    \usepackage{grffile} % extends the file name processing of package graphics 
                         % to support a larger range 
    % The hyperref package gives us a pdf with properly built
    % internal navigation ('pdf bookmarks' for the table of contents,
    % internal cross-reference links, web links for URLs, etc.)
    \usepackage{hyperref}
    \usepackage{longtable} % longtable support required by pandoc >1.10
    \usepackage{booktabs}  % table support for pandoc > 1.12.2
    \usepackage[inline]{enumitem} % IRkernel/repr support (it uses the enumerate* environment)
    \usepackage[normalem]{ulem} % ulem is needed to support strikethroughs (\sout)
                                % normalem makes italics be italics, not underlines
    

    
    
    % Colors for the hyperref package
    \definecolor{urlcolor}{rgb}{0,.145,.698}
    \definecolor{linkcolor}{rgb}{.71,0.21,0.01}
    \definecolor{citecolor}{rgb}{.12,.54,.11}

    % ANSI colors
    \definecolor{ansi-black}{HTML}{3E424D}
    \definecolor{ansi-black-intense}{HTML}{282C36}
    \definecolor{ansi-red}{HTML}{E75C58}
    \definecolor{ansi-red-intense}{HTML}{B22B31}
    \definecolor{ansi-green}{HTML}{00A250}
    \definecolor{ansi-green-intense}{HTML}{007427}
    \definecolor{ansi-yellow}{HTML}{DDB62B}
    \definecolor{ansi-yellow-intense}{HTML}{B27D12}
    \definecolor{ansi-blue}{HTML}{208FFB}
    \definecolor{ansi-blue-intense}{HTML}{0065CA}
    \definecolor{ansi-magenta}{HTML}{D160C4}
    \definecolor{ansi-magenta-intense}{HTML}{A03196}
    \definecolor{ansi-cyan}{HTML}{60C6C8}
    \definecolor{ansi-cyan-intense}{HTML}{258F8F}
    \definecolor{ansi-white}{HTML}{C5C1B4}
    \definecolor{ansi-white-intense}{HTML}{A1A6B2}

    % commands and environments needed by pandoc snippets
    % extracted from the output of `pandoc -s`
    \providecommand{\tightlist}{%
      \setlength{\itemsep}{0pt}\setlength{\parskip}{0pt}}
    \DefineVerbatimEnvironment{Highlighting}{Verbatim}{commandchars=\\\{\}}
    % Add ',fontsize=\small' for more characters per line
    \newenvironment{Shaded}{}{}
    \newcommand{\KeywordTok}[1]{\textcolor[rgb]{0.00,0.44,0.13}{\textbf{{#1}}}}
    \newcommand{\DataTypeTok}[1]{\textcolor[rgb]{0.56,0.13,0.00}{{#1}}}
    \newcommand{\DecValTok}[1]{\textcolor[rgb]{0.25,0.63,0.44}{{#1}}}
    \newcommand{\BaseNTok}[1]{\textcolor[rgb]{0.25,0.63,0.44}{{#1}}}
    \newcommand{\FloatTok}[1]{\textcolor[rgb]{0.25,0.63,0.44}{{#1}}}
    \newcommand{\CharTok}[1]{\textcolor[rgb]{0.25,0.44,0.63}{{#1}}}
    \newcommand{\StringTok}[1]{\textcolor[rgb]{0.25,0.44,0.63}{{#1}}}
    \newcommand{\CommentTok}[1]{\textcolor[rgb]{0.38,0.63,0.69}{\textit{{#1}}}}
    \newcommand{\OtherTok}[1]{\textcolor[rgb]{0.00,0.44,0.13}{{#1}}}
    \newcommand{\AlertTok}[1]{\textcolor[rgb]{1.00,0.00,0.00}{\textbf{{#1}}}}
    \newcommand{\FunctionTok}[1]{\textcolor[rgb]{0.02,0.16,0.49}{{#1}}}
    \newcommand{\RegionMarkerTok}[1]{{#1}}
    \newcommand{\ErrorTok}[1]{\textcolor[rgb]{1.00,0.00,0.00}{\textbf{{#1}}}}
    \newcommand{\NormalTok}[1]{{#1}}
    
    % Additional commands for more recent versions of Pandoc
    \newcommand{\ConstantTok}[1]{\textcolor[rgb]{0.53,0.00,0.00}{{#1}}}
    \newcommand{\SpecialCharTok}[1]{\textcolor[rgb]{0.25,0.44,0.63}{{#1}}}
    \newcommand{\VerbatimStringTok}[1]{\textcolor[rgb]{0.25,0.44,0.63}{{#1}}}
    \newcommand{\SpecialStringTok}[1]{\textcolor[rgb]{0.73,0.40,0.53}{{#1}}}
    \newcommand{\ImportTok}[1]{{#1}}
    \newcommand{\DocumentationTok}[1]{\textcolor[rgb]{0.73,0.13,0.13}{\textit{{#1}}}}
    \newcommand{\AnnotationTok}[1]{\textcolor[rgb]{0.38,0.63,0.69}{\textbf{\textit{{#1}}}}}
    \newcommand{\CommentVarTok}[1]{\textcolor[rgb]{0.38,0.63,0.69}{\textbf{\textit{{#1}}}}}
    \newcommand{\VariableTok}[1]{\textcolor[rgb]{0.10,0.09,0.49}{{#1}}}
    \newcommand{\ControlFlowTok}[1]{\textcolor[rgb]{0.00,0.44,0.13}{\textbf{{#1}}}}
    \newcommand{\OperatorTok}[1]{\textcolor[rgb]{0.40,0.40,0.40}{{#1}}}
    \newcommand{\BuiltInTok}[1]{{#1}}
    \newcommand{\ExtensionTok}[1]{{#1}}
    \newcommand{\PreprocessorTok}[1]{\textcolor[rgb]{0.74,0.48,0.00}{{#1}}}
    \newcommand{\AttributeTok}[1]{\textcolor[rgb]{0.49,0.56,0.16}{{#1}}}
    \newcommand{\InformationTok}[1]{\textcolor[rgb]{0.38,0.63,0.69}{\textbf{\textit{{#1}}}}}
    \newcommand{\WarningTok}[1]{\textcolor[rgb]{0.38,0.63,0.69}{\textbf{\textit{{#1}}}}}
    
    
    % Define a nice break command that doesn't care if a line doesn't already
    % exist.
    \def\br{\hspace*{\fill} \\* }
    % Math Jax compatability definitions
    \def\gt{>}
    \def\lt{<}
    % Document parameters
    \title{namR}
    
    
    

    % Pygments definitions
    
\makeatletter
\def\PY@reset{\let\PY@it=\relax \let\PY@bf=\relax%
    \let\PY@ul=\relax \let\PY@tc=\relax%
    \let\PY@bc=\relax \let\PY@ff=\relax}
\def\PY@tok#1{\csname PY@tok@#1\endcsname}
\def\PY@toks#1+{\ifx\relax#1\empty\else%
    \PY@tok{#1}\expandafter\PY@toks\fi}
\def\PY@do#1{\PY@bc{\PY@tc{\PY@ul{%
    \PY@it{\PY@bf{\PY@ff{#1}}}}}}}
\def\PY#1#2{\PY@reset\PY@toks#1+\relax+\PY@do{#2}}

\expandafter\def\csname PY@tok@w\endcsname{\def\PY@tc##1{\textcolor[rgb]{0.73,0.73,0.73}{##1}}}
\expandafter\def\csname PY@tok@c\endcsname{\let\PY@it=\textit\def\PY@tc##1{\textcolor[rgb]{0.25,0.50,0.50}{##1}}}
\expandafter\def\csname PY@tok@cp\endcsname{\def\PY@tc##1{\textcolor[rgb]{0.74,0.48,0.00}{##1}}}
\expandafter\def\csname PY@tok@k\endcsname{\let\PY@bf=\textbf\def\PY@tc##1{\textcolor[rgb]{0.00,0.50,0.00}{##1}}}
\expandafter\def\csname PY@tok@kp\endcsname{\def\PY@tc##1{\textcolor[rgb]{0.00,0.50,0.00}{##1}}}
\expandafter\def\csname PY@tok@kt\endcsname{\def\PY@tc##1{\textcolor[rgb]{0.69,0.00,0.25}{##1}}}
\expandafter\def\csname PY@tok@o\endcsname{\def\PY@tc##1{\textcolor[rgb]{0.40,0.40,0.40}{##1}}}
\expandafter\def\csname PY@tok@ow\endcsname{\let\PY@bf=\textbf\def\PY@tc##1{\textcolor[rgb]{0.67,0.13,1.00}{##1}}}
\expandafter\def\csname PY@tok@nb\endcsname{\def\PY@tc##1{\textcolor[rgb]{0.00,0.50,0.00}{##1}}}
\expandafter\def\csname PY@tok@nf\endcsname{\def\PY@tc##1{\textcolor[rgb]{0.00,0.00,1.00}{##1}}}
\expandafter\def\csname PY@tok@nc\endcsname{\let\PY@bf=\textbf\def\PY@tc##1{\textcolor[rgb]{0.00,0.00,1.00}{##1}}}
\expandafter\def\csname PY@tok@nn\endcsname{\let\PY@bf=\textbf\def\PY@tc##1{\textcolor[rgb]{0.00,0.00,1.00}{##1}}}
\expandafter\def\csname PY@tok@ne\endcsname{\let\PY@bf=\textbf\def\PY@tc##1{\textcolor[rgb]{0.82,0.25,0.23}{##1}}}
\expandafter\def\csname PY@tok@nv\endcsname{\def\PY@tc##1{\textcolor[rgb]{0.10,0.09,0.49}{##1}}}
\expandafter\def\csname PY@tok@no\endcsname{\def\PY@tc##1{\textcolor[rgb]{0.53,0.00,0.00}{##1}}}
\expandafter\def\csname PY@tok@nl\endcsname{\def\PY@tc##1{\textcolor[rgb]{0.63,0.63,0.00}{##1}}}
\expandafter\def\csname PY@tok@ni\endcsname{\let\PY@bf=\textbf\def\PY@tc##1{\textcolor[rgb]{0.60,0.60,0.60}{##1}}}
\expandafter\def\csname PY@tok@na\endcsname{\def\PY@tc##1{\textcolor[rgb]{0.49,0.56,0.16}{##1}}}
\expandafter\def\csname PY@tok@nt\endcsname{\let\PY@bf=\textbf\def\PY@tc##1{\textcolor[rgb]{0.00,0.50,0.00}{##1}}}
\expandafter\def\csname PY@tok@nd\endcsname{\def\PY@tc##1{\textcolor[rgb]{0.67,0.13,1.00}{##1}}}
\expandafter\def\csname PY@tok@s\endcsname{\def\PY@tc##1{\textcolor[rgb]{0.73,0.13,0.13}{##1}}}
\expandafter\def\csname PY@tok@sd\endcsname{\let\PY@it=\textit\def\PY@tc##1{\textcolor[rgb]{0.73,0.13,0.13}{##1}}}
\expandafter\def\csname PY@tok@si\endcsname{\let\PY@bf=\textbf\def\PY@tc##1{\textcolor[rgb]{0.73,0.40,0.53}{##1}}}
\expandafter\def\csname PY@tok@se\endcsname{\let\PY@bf=\textbf\def\PY@tc##1{\textcolor[rgb]{0.73,0.40,0.13}{##1}}}
\expandafter\def\csname PY@tok@sr\endcsname{\def\PY@tc##1{\textcolor[rgb]{0.73,0.40,0.53}{##1}}}
\expandafter\def\csname PY@tok@ss\endcsname{\def\PY@tc##1{\textcolor[rgb]{0.10,0.09,0.49}{##1}}}
\expandafter\def\csname PY@tok@sx\endcsname{\def\PY@tc##1{\textcolor[rgb]{0.00,0.50,0.00}{##1}}}
\expandafter\def\csname PY@tok@m\endcsname{\def\PY@tc##1{\textcolor[rgb]{0.40,0.40,0.40}{##1}}}
\expandafter\def\csname PY@tok@gh\endcsname{\let\PY@bf=\textbf\def\PY@tc##1{\textcolor[rgb]{0.00,0.00,0.50}{##1}}}
\expandafter\def\csname PY@tok@gu\endcsname{\let\PY@bf=\textbf\def\PY@tc##1{\textcolor[rgb]{0.50,0.00,0.50}{##1}}}
\expandafter\def\csname PY@tok@gd\endcsname{\def\PY@tc##1{\textcolor[rgb]{0.63,0.00,0.00}{##1}}}
\expandafter\def\csname PY@tok@gi\endcsname{\def\PY@tc##1{\textcolor[rgb]{0.00,0.63,0.00}{##1}}}
\expandafter\def\csname PY@tok@gr\endcsname{\def\PY@tc##1{\textcolor[rgb]{1.00,0.00,0.00}{##1}}}
\expandafter\def\csname PY@tok@ge\endcsname{\let\PY@it=\textit}
\expandafter\def\csname PY@tok@gs\endcsname{\let\PY@bf=\textbf}
\expandafter\def\csname PY@tok@gp\endcsname{\let\PY@bf=\textbf\def\PY@tc##1{\textcolor[rgb]{0.00,0.00,0.50}{##1}}}
\expandafter\def\csname PY@tok@go\endcsname{\def\PY@tc##1{\textcolor[rgb]{0.53,0.53,0.53}{##1}}}
\expandafter\def\csname PY@tok@gt\endcsname{\def\PY@tc##1{\textcolor[rgb]{0.00,0.27,0.87}{##1}}}
\expandafter\def\csname PY@tok@err\endcsname{\def\PY@bc##1{\setlength{\fboxsep}{0pt}\fcolorbox[rgb]{1.00,0.00,0.00}{1,1,1}{\strut ##1}}}
\expandafter\def\csname PY@tok@kc\endcsname{\let\PY@bf=\textbf\def\PY@tc##1{\textcolor[rgb]{0.00,0.50,0.00}{##1}}}
\expandafter\def\csname PY@tok@kd\endcsname{\let\PY@bf=\textbf\def\PY@tc##1{\textcolor[rgb]{0.00,0.50,0.00}{##1}}}
\expandafter\def\csname PY@tok@kn\endcsname{\let\PY@bf=\textbf\def\PY@tc##1{\textcolor[rgb]{0.00,0.50,0.00}{##1}}}
\expandafter\def\csname PY@tok@kr\endcsname{\let\PY@bf=\textbf\def\PY@tc##1{\textcolor[rgb]{0.00,0.50,0.00}{##1}}}
\expandafter\def\csname PY@tok@bp\endcsname{\def\PY@tc##1{\textcolor[rgb]{0.00,0.50,0.00}{##1}}}
\expandafter\def\csname PY@tok@fm\endcsname{\def\PY@tc##1{\textcolor[rgb]{0.00,0.00,1.00}{##1}}}
\expandafter\def\csname PY@tok@vc\endcsname{\def\PY@tc##1{\textcolor[rgb]{0.10,0.09,0.49}{##1}}}
\expandafter\def\csname PY@tok@vg\endcsname{\def\PY@tc##1{\textcolor[rgb]{0.10,0.09,0.49}{##1}}}
\expandafter\def\csname PY@tok@vi\endcsname{\def\PY@tc##1{\textcolor[rgb]{0.10,0.09,0.49}{##1}}}
\expandafter\def\csname PY@tok@vm\endcsname{\def\PY@tc##1{\textcolor[rgb]{0.10,0.09,0.49}{##1}}}
\expandafter\def\csname PY@tok@sa\endcsname{\def\PY@tc##1{\textcolor[rgb]{0.73,0.13,0.13}{##1}}}
\expandafter\def\csname PY@tok@sb\endcsname{\def\PY@tc##1{\textcolor[rgb]{0.73,0.13,0.13}{##1}}}
\expandafter\def\csname PY@tok@sc\endcsname{\def\PY@tc##1{\textcolor[rgb]{0.73,0.13,0.13}{##1}}}
\expandafter\def\csname PY@tok@dl\endcsname{\def\PY@tc##1{\textcolor[rgb]{0.73,0.13,0.13}{##1}}}
\expandafter\def\csname PY@tok@s2\endcsname{\def\PY@tc##1{\textcolor[rgb]{0.73,0.13,0.13}{##1}}}
\expandafter\def\csname PY@tok@sh\endcsname{\def\PY@tc##1{\textcolor[rgb]{0.73,0.13,0.13}{##1}}}
\expandafter\def\csname PY@tok@s1\endcsname{\def\PY@tc##1{\textcolor[rgb]{0.73,0.13,0.13}{##1}}}
\expandafter\def\csname PY@tok@mb\endcsname{\def\PY@tc##1{\textcolor[rgb]{0.40,0.40,0.40}{##1}}}
\expandafter\def\csname PY@tok@mf\endcsname{\def\PY@tc##1{\textcolor[rgb]{0.40,0.40,0.40}{##1}}}
\expandafter\def\csname PY@tok@mh\endcsname{\def\PY@tc##1{\textcolor[rgb]{0.40,0.40,0.40}{##1}}}
\expandafter\def\csname PY@tok@mi\endcsname{\def\PY@tc##1{\textcolor[rgb]{0.40,0.40,0.40}{##1}}}
\expandafter\def\csname PY@tok@il\endcsname{\def\PY@tc##1{\textcolor[rgb]{0.40,0.40,0.40}{##1}}}
\expandafter\def\csname PY@tok@mo\endcsname{\def\PY@tc##1{\textcolor[rgb]{0.40,0.40,0.40}{##1}}}
\expandafter\def\csname PY@tok@ch\endcsname{\let\PY@it=\textit\def\PY@tc##1{\textcolor[rgb]{0.25,0.50,0.50}{##1}}}
\expandafter\def\csname PY@tok@cm\endcsname{\let\PY@it=\textit\def\PY@tc##1{\textcolor[rgb]{0.25,0.50,0.50}{##1}}}
\expandafter\def\csname PY@tok@cpf\endcsname{\let\PY@it=\textit\def\PY@tc##1{\textcolor[rgb]{0.25,0.50,0.50}{##1}}}
\expandafter\def\csname PY@tok@c1\endcsname{\let\PY@it=\textit\def\PY@tc##1{\textcolor[rgb]{0.25,0.50,0.50}{##1}}}
\expandafter\def\csname PY@tok@cs\endcsname{\let\PY@it=\textit\def\PY@tc##1{\textcolor[rgb]{0.25,0.50,0.50}{##1}}}

\def\PYZbs{\char`\\}
\def\PYZus{\char`\_}
\def\PYZob{\char`\{}
\def\PYZcb{\char`\}}
\def\PYZca{\char`\^}
\def\PYZam{\char`\&}
\def\PYZlt{\char`\<}
\def\PYZgt{\char`\>}
\def\PYZsh{\char`\#}
\def\PYZpc{\char`\%}
\def\PYZdl{\char`\$}
\def\PYZhy{\char`\-}
\def\PYZsq{\char`\'}
\def\PYZdq{\char`\"}
\def\PYZti{\char`\~}
% for compatibility with earlier versions
\def\PYZat{@}
\def\PYZlb{[}
\def\PYZrb{]}
\makeatother


    % Exact colors from NB
    \definecolor{incolor}{rgb}{0.0, 0.0, 0.5}
    \definecolor{outcolor}{rgb}{0.545, 0.0, 0.0}



    
    % Prevent overflowing lines due to hard-to-break entities
    \sloppy 
    % Setup hyperref package
    \hypersetup{
      breaklinks=true,  % so long urls are correctly broken across lines
      colorlinks=true,
      urlcolor=urlcolor,
      linkcolor=linkcolor,
      citecolor=citecolor,
      }
    % Slightly bigger margins than the latex defaults
    
    \geometry{verbose,tmargin=1in,bmargin=1in,lmargin=1in,rmargin=1in}
    
    

    \begin{document}
    
    
    \maketitle
    
    

    
    \hypertarget{database}{%
\section{Database}\label{database}}

    My first choice was MongoDB, since it would be fast and horizontal
scalable. But found out that MongoDB has not necessary features and
functions to calculate distance between different geometries. Then
choice laid on PostGIS, since it has all the necessary tools, except
some issues while dealing with big amounts of data, which will be
covered through this report.

    \hypertarget{data-ingestion}{%
\section{Data Ingestion}\label{data-ingestion}}

    If it is single import of data to PostGIS, we can easily upload through
PgAdmin GUI, or with single line of code. Nevertheless, I designed
Python code for both import style upload or batch style upload.

    \hypertarget{data-visualization}{%
\section{Data Visualization}\label{data-visualization}}

    \hypertarget{data-cleaning}{%
\subsubsection{Data cleaning}\label{data-cleaning}}

Since all the data in dataset was pretty clear, I just made PostGIS
validation with st\_MakeValid. The function which attempts to create a
valid representation of a given invalid geometry without losing any of
the input vertices.

\begin{Shaded}
\begin{Highlighting}[]
\KeywordTok{UPDATE}\NormalTok{ buildings}
\KeywordTok{SET}\NormalTok{ location=ST_MakeValid(location);}

\KeywordTok{UPDATE}\NormalTok{ trees}
\KeywordTok{SET}\NormalTok{ location=ST_MakeValid(location);}
\end{Highlighting}
\end{Shaded}

    \hypertarget{data-visualization}{%
\subsubsection{Data Visualization}\label{data-visualization}}

    I was not reinventing the wheel and used PgAdmin's GUI tools to
visualize data

    \begin{Verbatim}[commandchars=\\\{\}]
{\color{incolor}In [{\color{incolor}12}]:} \PY{k+kn}{from} \PY{n+nn}{IPython}\PY{n+nn}{.}\PY{n+nn}{display} \PY{k}{import} \PY{n}{Image}
         \PY{n}{Image}\PY{p}{(}\PY{l+s+s2}{\PYZdq{}}\PY{l+s+s2}{/Users/meirkhan/Desktop/buildings.png}\PY{l+s+s2}{\PYZdq{}}\PY{p}{)}
\end{Verbatim}

\texttt{\color{outcolor}Out[{\color{outcolor}12}]:}
    
    \begin{center}
    \adjustimage{max size={0.9\linewidth}{0.9\paperheight}}{output_8_0.png}
    \end{center}
    { \hspace*{\fill} \\}
    

    \begin{Verbatim}[commandchars=\\\{\}]
{\color{incolor}In [{\color{incolor}11}]:} \PY{n}{Image}\PY{p}{(}\PY{l+s+s2}{\PYZdq{}}\PY{l+s+s2}{/Users/meirkhan/Desktop/trees.png}\PY{l+s+s2}{\PYZdq{}}\PY{p}{)}
\end{Verbatim}

\texttt{\color{outcolor}Out[{\color{outcolor}11}]:}
    
    \begin{center}
    \adjustimage{max size={0.9\linewidth}{0.9\paperheight}}{output_9_0.png}
    \end{center}
    { \hspace*{\fill} \\}
    

    \hypertarget{finding-closest-tree}{%
\subsection{Finding closest tree}\label{finding-closest-tree}}

\begin{Shaded}
\begin{Highlighting}[]
\KeywordTok{WITH} 
\NormalTok{cte }\KeywordTok{as}\NormalTok{ (}
        \KeywordTok{SELECT} 
          \KeywordTok{distinct} \KeywordTok{on}\NormalTok{ (b.id)  b.id building_id,}
\NormalTok{          t.id tree_id}
        \KeywordTok{FROM}\NormalTok{ buildings }\KeywordTok{as}\NormalTok{ b,}
\NormalTok{             treees }\KeywordTok{as}\NormalTok{ t}
        \KeywordTok{ORDER} \KeywordTok{BY}\NormalTok{ b.id, }
\NormalTok{                 St_distance(b.location:}\CharTok{:geometry}\NormalTok{, t.location:}\CharTok{:geometry}\NormalTok{)}
\NormalTok{       )}
\KeywordTok{SELECT} 
\NormalTok{  cte.building_id,}
\NormalTok{  cte.tree_id,}
\NormalTok{  St_distance(b.location, t.location)}
\KeywordTok{FROM}\NormalTok{ cte, }
\NormalTok{     buildings b, }
\NormalTok{     trees t}
\KeywordTok{WHERE}\NormalTok{ b.id = cte.building_id}
  \KeywordTok{AND}\NormalTok{ t.id = cte.tree_id;}
\end{Highlighting}
\end{Shaded}

    \begin{Verbatim}[commandchars=\\\{\}]
{\color{incolor}In [{\color{incolor}13}]:} \PY{n}{Image}\PY{p}{(}\PY{l+s+s2}{\PYZdq{}}\PY{l+s+s2}{/Users/meirkhan/Desktop/1st.png}\PY{l+s+s2}{\PYZdq{}}\PY{p}{)}
         \PY{c+c1}{\PYZsh{} ![title](/Users/meirkhan/Desktop/pic.png)}
\end{Verbatim}

\texttt{\color{outcolor}Out[{\color{outcolor}13}]:}
    
    \begin{center}
    \adjustimage{max size={0.9\linewidth}{0.9\paperheight}}{output_11_0.png}
    \end{center}
    { \hspace*{\fill} \\}
    

    \hypertarget{finding-count-of-trees-in-100-m}{%
\subsection{Finding count of trees in 100
m}\label{finding-count-of-trees-in-100-m}}

\begin{Shaded}
\begin{Highlighting}[]
\KeywordTok{SELECT}\NormalTok{ b.id, }
\NormalTok{      (}\KeywordTok{CASE} \KeywordTok{WHEN}\NormalTok{ x.cnt }\KeywordTok{is} \KeywordTok{NULL} \KeywordTok{THEN} \DecValTok{0} \KeywordTok{ELSE}\NormalTok{ x.cnt }\KeywordTok{END}\NormalTok{) cnt}
\KeywordTok{FROM}\NormalTok{ buildings b }
\KeywordTok{LEFT} \KeywordTok{JOIN}
\NormalTok{      (}
        \KeywordTok{SELECT}\NormalTok{ b.id building_id, }\FunctionTok{count}\NormalTok{(b.id) cnt }
        \KeywordTok{FROM}\NormalTok{ buildings b, }
\NormalTok{             trees t}
        \KeywordTok{WHERE}\NormalTok{ st_dwithin(b.location:}\CharTok{:geography}\NormalTok{, t.location:}\CharTok{:geography}\NormalTok{, }\DecValTok{100}\NormalTok{)}
        \KeywordTok{GROUP} \KeywordTok{BY}\NormalTok{ b.id}
\NormalTok{      ) x}
\KeywordTok{ON}\NormalTok{ b.id = x.building_id;}
\end{Highlighting}
\end{Shaded}

    \begin{Verbatim}[commandchars=\\\{\}]
{\color{incolor}In [{\color{incolor}9}]:} \PY{k+kn}{from} \PY{n+nn}{IPython}\PY{n+nn}{.}\PY{n+nn}{display} \PY{k}{import} \PY{n}{Image}
        \PY{n}{Image}\PY{p}{(}\PY{l+s+s2}{\PYZdq{}}\PY{l+s+s2}{/Users/meirkhan/Desktop/2nd.png}\PY{l+s+s2}{\PYZdq{}}\PY{p}{)}
\end{Verbatim}

\texttt{\color{outcolor}Out[{\color{outcolor}9}]:}
    
    \begin{center}
    \adjustimage{max size={0.9\linewidth}{0.9\paperheight}}{output_13_0.png}
    \end{center}
    { \hspace*{\fill} \\}
    

    As we infer from both tables, best performance is obtained when
operating with \textbf{spatial indexes, geography datatype and
SRID:2163} (US National Atlas Equal Area).

Even though it is fastest SRID recommended by many sources, it's results
are not that accurate. I assumed in this task accuracy is very
important, therefore I used only \textbf{SRID:4326} on final results.

    \hypertarget{aggregated-table}{%
\subsection{Aggregated table}\label{aggregated-table}}

On whole dataset this query performs in nearly 5 minutes

    \begin{Shaded}
\begin{Highlighting}[]
\KeywordTok{CREATE}\NormalTok{ aggregated_table }\KeywordTok{AS}\NormalTok{ ( }
\KeywordTok{WITH} 
\NormalTok{cte1 }\KeywordTok{AS} 
\NormalTok{        ( }
          \KeywordTok{SELECT} \KeywordTok{DISTINCT} \KeywordTok{ON}\NormalTok{ (b.id) b.id building_id, }
\NormalTok{          t.id tree_id }
          \KeywordTok{FROM}\NormalTok{ buildings b, }
\NormalTok{               trees     t }
          \KeywordTok{ORDER} \KeywordTok{BY}\NormalTok{ b.id, }
\NormalTok{                   st_distance(b.location:}\CharTok{:geometry}\NormalTok{, t.location:}\CharTok{:geometry}\NormalTok{) }
\NormalTok{        ), }
\NormalTok{cte2 }\KeywordTok{AS}
\NormalTok{        ( }
          \KeywordTok{SELECT}\NormalTok{ b.id, ( }
                    \KeywordTok{CASE} 
                      \KeywordTok{WHEN}\NormalTok{ x.cnt }\KeywordTok{IS} \KeywordTok{NULL} 
                      \KeywordTok{THEN} \DecValTok{0} 
                      \KeywordTok{ELSE}\NormalTok{ x.cnt }
                    \KeywordTok{END}\NormalTok{) cnt }
          \KeywordTok{FROM}\NormalTok{      buildings1 b }
          \KeywordTok{LEFT} \KeywordTok{JOIN} 
\NormalTok{                    ( }
                      \KeywordTok{SELECT}\NormalTok{   b.id building_id, }
                               \FunctionTok{count}\NormalTok{(b.id) cnt }
                      \KeywordTok{FROM}\NormalTok{  buildings b, }
\NormalTok{                            trees t }
                      \KeywordTok{WHERE}\NormalTok{ st_dwithin(b.location:}\CharTok{:geography}\NormalTok{, t.location:}\CharTok{:geography}\NormalTok{, }\DecValTok{100}\NormalTok{)}
                      \KeywordTok{GROUP} \KeywordTok{BY}\NormalTok{ b.id}
\NormalTok{                    ) x }
          \KeywordTok{ON}\NormalTok{ b.id = x.building_id }
\NormalTok{        )}
\KeywordTok{SELECT}\NormalTok{ cte1.building_id, }
\NormalTok{       cte1.tree_id, }
\NormalTok{       St_distance(b.location, t.location) dist, }
\NormalTok{       cte2.cnt }
\KeywordTok{FROM}\NormalTok{   cte1, }
\NormalTok{       cte2, }
\NormalTok{       buildings1 b, }
\NormalTok{       trees1 t }
\KeywordTok{WHERE}\NormalTok{  b.id = cte1.building_id }
\KeywordTok{AND}\NormalTok{    t.id = cte1.tree_id }
\KeywordTok{AND}\NormalTok{    cte1.building_id=cte2.id);}
\end{Highlighting}
\end{Shaded}

    \hypertarget{pipeline-for-data-integration}{%
\subsection{Pipeline for Data
Integration}\label{pipeline-for-data-integration}}

    \begin{Verbatim}[commandchars=\\\{\}]
{\color{incolor}In [{\color{incolor}14}]:} \PY{n}{Image}\PY{p}{(}\PY{l+s+s2}{\PYZdq{}}\PY{l+s+s2}{/Users/meirkhan/Desktop/pipe.png}\PY{l+s+s2}{\PYZdq{}}\PY{p}{)}
\end{Verbatim}

\texttt{\color{outcolor}Out[{\color{outcolor}14}]:}
    
    \begin{center}
    \adjustimage{max size={0.9\linewidth}{0.9\paperheight}}{output_18_0.png}
    \end{center}
    { \hspace*{\fill} \\}
    

    As one option to integration of newly arrived data, we can define two
stored procedures, for ``buildings'' and ``trees'' respectively.

\begin{itemize}
\tightlist
\item
  So that, new buildings data would be stored in temporary table, stored
  procedure would calculate nearest tree for them, count of trees in
  100m and insert it to aggregated table. Stored procedure is as below:
\end{itemize}

    \begin{Shaded}
\begin{Highlighting}[]

\KeywordTok{CREATE} \KeywordTok{OR} \KeywordTok{REPLACE} \KeywordTok{PROCEDURE}\NormalTok{ updateBuildings()}
\NormalTok{LANGUAGE plpgsql    }
\KeywordTok{AS}\NormalTok{ $$}
\KeywordTok{BEGIN}
    \KeywordTok{INSERT} \KeywordTok{INTO}\NormalTok{ aggregated_table(building_id, tree_id, dist, cnt,b_loc,t_loc) }
\KeywordTok{WITH}
\NormalTok{cte1 }\KeywordTok{as}\NormalTok{ (}
         \KeywordTok{SELECT} 
            \KeywordTok{distinct} \KeywordTok{on}\NormalTok{ (b.id)  b.id building_id,}
\NormalTok{            t.id tree_id,}
\NormalTok{            b.location b_loc,}
\NormalTok{            t.location t_loc}
        \KeywordTok{FROM}\NormalTok{ temp_buildings }\KeywordTok{as}\NormalTok{ b,}
\NormalTok{             trees }\KeywordTok{as}\NormalTok{ t}
        \KeywordTok{ORDER} \KeywordTok{BY}\NormalTok{ b.id, St_distance(b.location:}\CharTok{:geometry}\NormalTok{, t.location:}\CharTok{:geometry}\NormalTok{)  }
\NormalTok{        ),  }
\NormalTok{cte2 }\KeywordTok{as}\NormalTok{ (}
        \KeywordTok{SELECT}\NormalTok{ b.id, (}\KeywordTok{case} \KeywordTok{when}\NormalTok{ x.cnt }\KeywordTok{is} \KeywordTok{null} \KeywordTok{then} \DecValTok{0} \KeywordTok{else}\NormalTok{ x.cnt }\KeywordTok{end}\NormalTok{) cnt}
        \KeywordTok{FROM}\NormalTok{ b1 b }\KeywordTok{LEFT} \KeywordTok{JOIN}
\NormalTok{        (}
            \KeywordTok{select}\NormalTok{ b.id , }\FunctionTok{count}\NormalTok{(b.id) cnt }\KeywordTok{from}
\NormalTok{            temp_buildings b, }
\NormalTok{            trees t}
            \KeywordTok{WHERE}\NormalTok{ st_dwithin(b.location:}\CharTok{:geography}\NormalTok{, t.location:}\CharTok{:geography}\NormalTok{, }\DecValTok{100}\NormalTok{)}
            \KeywordTok{GROUP} \KeywordTok{BY}\NormalTok{ b.id}
\NormalTok{        ) x}
        \KeywordTok{ON}\NormalTok{ b.id = x.id}
\NormalTok{)}
 
\KeywordTok{SELECT} 
\NormalTok{    cte1.building_id,}
\NormalTok{    cte1.tree_id,}
\NormalTok{    St_distance(b.location, t.location) dist,}
\NormalTok{    cte2.cnt,}
\NormalTok{    cte1.b_loc,}
\NormalTok{    cte1.t_loc}
\KeywordTok{FROM}\NormalTok{ cte1, }
\NormalTok{      temp_buildings b, }
\NormalTok{      trees t, }
\NormalTok{      cte2    }
\KeywordTok{WHERE}\NormalTok{ b.id = cte1.building_id}
  \KeywordTok{AND}\NormalTok{ t.id = cte1.tree_id}
  \KeywordTok{AND}\NormalTok{ cte1.building_id=cte2.id;}
 
    \KeywordTok{COMMIT}\NormalTok{;}
\KeywordTok{END}\NormalTok{;}
\NormalTok{$$;}


\KeywordTok{call}\NormalTok{ updateBuildings();}

\end{Highlighting}
\end{Shaded}

    Respectively, as new trees data arrives, it will be stored in temporary
table, and unlike buildings data, stored procedure would be performing
UPDATES instead of INSERTS. So that, it will be calculated with joins on
buildings table and respective row in AGGREGATED table and\\
- Tree id and distance will be updated if tree is closer than old one -
Count of trees in 100m radius will increment by 1, if tree lays in that
radius

CODE:

    \begin{Shaded}
\begin{Highlighting}[]
\KeywordTok{CREATE} \KeywordTok{OR} \KeywordTok{REPLACE} \KeywordTok{PROCEDURE}\NormalTok{ updateTrees()}
\NormalTok{LANGUAGE plpgsql}
\KeywordTok{AS}\NormalTok{ $$}
\KeywordTok{DECLARE}
\NormalTok{    r1 }\DataTypeTok{record}\NormalTok{;}
\NormalTok{    r2 }\DataTypeTok{record}\NormalTok{;}
\NormalTok{    new_dist }\DataTypeTok{float}\NormalTok{;}
\NormalTok{    isInRadius }\DataTypeTok{boolean}\NormalTok{;}
\KeywordTok{BEGIN}
    \KeywordTok{FOR}\NormalTok{ r1 }\KeywordTok{IN} \KeywordTok{select}\NormalTok{ * }\KeywordTok{from}\NormalTok{ aggregated_table}
    \KeywordTok{LOOP}
        \KeywordTok{FOR}\NormalTok{ r2 }\KeywordTok{IN} \KeywordTok{SELECT}\NormalTok{ * }\KeywordTok{from}\NormalTok{ temp_tree_table}
        \KeywordTok{LOOP}
            \KeywordTok{IF}\NormalTok{ st_distance(r1.b_loc:}\CharTok{:geography}\NormalTok{,r2.location:}\CharTok{:geography}\NormalTok{) < r1.dist }\KeywordTok{then}
\NormalTok{                new_dist := st_distance(r1.b_loc:}\CharTok{:geography}\NormalTok{,r2.location:}\CharTok{:geography}\NormalTok{);                }
            \KeywordTok{END} \KeywordTok{IF}\NormalTok{;}
            \KeywordTok{IF}\NormalTok{ st_dwithin(r1.b_loc:}\CharTok{:geography}\NormalTok{, r2.location:}\CharTok{:geography}\NormalTok{, }\DecValTok{100}\NormalTok{) }\KeywordTok{then}
\NormalTok{                isInRadius := }\KeywordTok{true}\NormalTok{;}
            \KeywordTok{END} \KeywordTok{IF}\NormalTok{;}
        \KeywordTok{END} \KeywordTok{LOOP}\NormalTok{;}
        \KeywordTok{UPDATE}\NormalTok{ aggregated_table }
        \KeywordTok{SET}\NormalTok{ dist = new_dist }
        \KeywordTok{WHERE}\NormalTok{ r1.building_id = r2.id;}
        \KeywordTok{IF}\NormalTok{ isInRadius = }\KeywordTok{true} \KeywordTok{then}
            \KeywordTok{UPDATE}\NormalTok{ aggregated_table }
            \KeywordTok{SET}\NormalTok{ cnt = }\DecValTok{1111} 
            \KeywordTok{WHERE}\NormalTok{ r1.building_id = r2.id;}
        \KeywordTok{END} \KeywordTok{IF}\NormalTok{;}
    \KeywordTok{END} \KeywordTok{LOOP}\NormalTok{;}
    \KeywordTok{COMMIT}\NormalTok{;}
\KeywordTok{END}\NormalTok{;}
\NormalTok{$$;}

\KeywordTok{call}\NormalTok{ updateTrees();}
\end{Highlighting}
\end{Shaded}

    Other than that, it is clear that we cannot handle big amount of data by
only user side tricks like index algorithms. Hence, I would suggest
implementing some server side improvements, like parallelism and try to
implement column store database.

And after some research, I found out that it is possible scale Postgres
horizontally with \textbf{Citus.} Citus solves the horizontal
partitioning problem in a clever way. Citus realized if they made far
more logical partitions of tables than physical nodes, they could - put
copies of partitions on multiple physical nodes for redundancy, so
losing one node never caused data loss, and - rebalance to new nodes by
moving logical partitions automatically between nodes, which is far
easier to automate than moving rows between nodes.


    % Add a bibliography block to the postdoc
    
    
    
    \end{document}
